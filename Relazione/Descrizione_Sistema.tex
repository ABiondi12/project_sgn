
\section{Descrizione del sistema}
 
Il veicolo, per gli amici e i lettori Charlie, \`e basato su un Crawler RC, una piattaforma meccanica radio--comandata, su cui sono stati installati dei sensori e delle schede elettroniche. 

\textbf{FARE IMMAGINE CONCETTUALE}

A bordo si trovano quindi due unit\`a centrali:
\begin{itemize}
	\item un Raspberry Pi 4 (8Gb Ram), con sistema operativo Linux 18.04 su cui viene eseguito Robot Operating System (ROS)
	\item una scheda STM32F407 su cui è implementato il sistema di guida e alcuni filtraggi
\end{itemize}

Come sensori sono presenti:
\begin{itemize}
	\item Lidar Slamtec RPLIDAR-A3
	
	\item due tag del sistema UWB creato da Pozyx che dialogano con 4 anchors disposte nell'ambiente
\end{itemize}





Per connettere e alimentare quanto qui sopra \`e stato installato:
\begin{itemize}
	\item una custom pcb 
	
	\item USB-HUB alimentato, che ci permette di utilizzare ulteriori porte usb senza far affidamento al Raspberry Pi per la loro alimentazione (che risulta inefficace per alimentare il lidar)
\end{itemize}