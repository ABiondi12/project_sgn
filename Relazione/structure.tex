\documentclass[
12pt, % Main document font size
a4paper, % Paper type, use 'letterpaper' for US Letter paper
oneside,
% One page layout (no page indentation)
%twoside, % Two page layout (page indentation for binding and different headers)
%hidelinks, % Hide links like hyperref
%headinclude,footinclude, % Extra spacing for the header and footer
%BCOR5mm, % Binding correction
]{article}



\setlength{\parindent}{5pt} %Modifica la dimensione dell'indentazione dei paragrafi



%%%%% Packages
%\usepackage [english]{babel}
\usepackage [italian]{babel}

\usepackage{enumitem} % migliori enumerate, ad esempio \begin{enumerate}[label=\Alph*]

%per far chiudere e aprire " "
%\usepackage [autostyle, english = american]{csquotes}
%\MakeOuterQuote{"}

\usepackage{hyphenat}
\begin{hyphenrules}{english}
\hyphenation{Fortran}
\hyphenation{spa-sti-ci-ty} 
\end{hyphenrules}

\usepackage[section]{placeins} %Also one can consider the placeins package and the \FloatBarrier command that prevents figures from floating any further. If figures are supposed to remain in the section, one can load

\usepackage{tabularx} %advance tables, like wrap text inside cell
\usepackage{longtable}

\usepackage{graphicx}
%\usepackage{subfig}
\usepackage{xcolor} %Colori Aggiuntiv
\usepackage[colorlinks = true,
	linkcolor = teal,
	urlcolor  = blue,
	citecolor = teal,
	anchorcolor = blue]{hyperref} %Hyperref
	
% listings for code	
\usepackage{listings}
\usepackage{color}
\definecolor{dkgreen}{rgb}{0,0.8,0}
\definecolor{gray}{rgb}{0.5,0.5,0.5}
\definecolor{mauve}{rgb}{0.58,0,0.82}

\lstset{frame=shadowbox,
  language=Matlab,
  aboveskip=3mm,
  belowskip=3mm,
  showstringspaces=false,
  columns=flexible,
  basicstyle={\small\ttfamily},
  numbers=none,
  numberstyle=\tiny\color{gray},
  keywordstyle=\color{blue},
  commentstyle=\color{dkgreen},
  stringstyle=\color{mauve},
  breaklines=true,
  breakatwhitespace=true,
  tabsize=3
}


\usepackage{url}

%\usepackage{lipsum} %Crea dummy text

\usepackage{amsmath,amssymb,amscd,amsfonts,amsthm} %Roba matematica

%bibliografia
%\usepackage[shortalphabetic,abbrev]{amsrefs}
%\usepackage[author-year, y2k]{amsrefs}
\usepackage[alphabetic]{amsrefs}
\usepackage{mathtools}

\usepackage{physics}

\usepackage{siunitx} %database unita` di misura, \si{ \kg \per \second},  \SI{20}{ \kg \per \second}
\sisetup{
	round-mode          = places,
	round-precision     = 2,
}

\usepackage{gensymb} %simboli extra, come \degree

\usepackage{tikz}
	\usetikzlibrary{shapes,arrows}
	\tikzstyle{block} = [draw, fill=white, rectangle, 
	minimum height=3em, minimum width=6em]
	\tikzstyle{sum} = [draw, fill=white, circle, node distance=1cm]
	\tikzstyle{input} = [coordinate]
	\tikzstyle{output} = [coordinate]
	\tikzstyle{pinstyle} = [pin edge={to-,thin,black}]
	\usepackage{tikzscale}

\usepackage{pgfplots}
	\pgfplotsset{width=7cm,compat=1.14}
	%\pgfplotsset{compat=newest} % Allows to place the legend below plot
	\pgfplotsset{every minor tick/.append style={thin}}  % applies only to minor ticks,
	\usepgfplotslibrary{units} % Allows to enter the units nicely
	%\usepgfplotslibrary{external} 
	%\tikzexternalize
	%\tikzset{external/force remake}
	\pgfkeys{/pgf/number format/.cd,1000 sep={\,}}

\usepackage{afterpage}

\usepackage{comment} % inserisce la funzionalita` \begin{comment}
\usepackage{wrapfig}
\usepackage{vmargin}
\setmarginsrb  {25mm}  % left margin
	{ 5mm}  % top margin
	{25mm}  % right margin
	{15mm}  % bottom margin
	{10mm}  % head height
	{15mm}  % head sep
	{10mm}  % foot height
	{15mm}  % foot sep



% permettono l'uso di subfigures
\usepackage{caption}
\usepackage{subcaption}


%%%%%%%%%%%%%%%%%%%%%%%%%%%%%%%%%%%%%%%%%%%%%%%%%%%%%%%%%%
%%%%% Comandi homebrewed


%% comando per fare matrix con linee verticali. Esempio:
%\[
%\begin{pmatrix}[cc|c]
%  1 & 2 & 3\\
%  4 & 5 & 9
%\end{pmatrix}
%\]
\makeatletter
\renewcommand*\env@matrix[1][*\c@MaxMatrixCols c]{%
  \hskip -\arraycolsep
  \let\@ifnextchar\new@ifnextchar
  \array{#1}}
\makeatother


%% renew comando paragraph per far andare a capo dopo \paragraph
\newcommand{\myparagraph}[1]{\paragraph{#1}\mbox{}\\}

\newcommand{\bs}{\textbackslash}

%% BLANKPAGE
%per creare pagine vuote
\newcommand\blankpage{%
	\null
	\thispagestyle{empty}%
	%\addtocounter{page}{-1}%
	\newpage}

%% RNUM 
% porta in numeri romani il unmero
\newcommand{\RNum}[1]{\uppercase\expandafter{\romannumeral #1\relax}}

%% AT
% Comando per le derivate calcolate nel punto. Esempio: \dv{\text{Im}T(j \omega)}{t} \at[\Big]{\omega = \omega_1}
\newcommand{\at}[2][]{#1|_{#2}}

%% Comando per valori assoluti che scalano
%\DeclarePairedDelimiter\abs{\lvert}{\rvert}%
%\DeclarePairedDelimiter\norm{\lVert}{\rVert}%
% Swap the definition of \abs* and \norm*, so that \abs
% and \norm resizes the size of the brackets, and the 
% starred version does not.
\makeatletter
\let\oldabs\abs
\def\abs{\@ifstar{\oldabs}{\oldabs*}}
%
%\let\oldnorm\norm
%\def\norm{\@ifstar{\oldnorm}{\oldnorm*}}
\makeatother

%% CEILING and FLOOR
\DeclarePairedDelimiter\ceil{\lceil}{\rceil}
\DeclarePairedDelimiter\floor{\lfloor}{\rfloor}

%% Prova theorem
\newtheorem{theorem}{Theorem}[section]
\newtheorem{lemma}[theorem]{Lemma}
\newtheorem{proposition}[theorem]{Proposition}
\newtheorem{corollary}[theorem]{Corollary}

\newenvironment{Proof}[1][Proof]{\begin{trivlist}
		\item[\hskip \labelsep {\bfseries #1}]}{\end{trivlist}}
\newenvironment{definition}[1][Definition]{\begin{trivlist}
		\item[\hskip \labelsep {\bfseries #1}]}{\end{trivlist}}
\newenvironment{example}[1][Example]{\begin{trivlist}
		\item[\hskip \labelsep {\bfseries #1}]}{\end{trivlist}}
\newenvironment{remark}[1][Remark]{\begin{trivlist}
		\item[\hskip \labelsep {\bfseries #1}]}{\end{trivlist}}

%\newcommand{\qed}{\nobreak \ifvmode \relax \else
%	\ifdim\lastskip<1.5em \hskip-\lastskip
%	\hskip1.5em plus0em minus0.5em \fi \nobreak
%	\vrule height0.75em width0.5em depth0.25em\fi}

%% \xoverline
% simile a \bar ma copre tutta il width della lettera
\makeatletter
\newsavebox\myboxA
\newsavebox\myboxB
\newlength\mylenA

\newcommand*\xoverline[2][0.75]{%
    \sbox{\myboxA}{$\m@th#2$}%
    \setbox\myboxB\null% Phantom box
    \ht\myboxB=\ht\myboxA%
    \dp\myboxB=\dp\myboxA%
    \wd\myboxB=#1\wd\myboxA% Scale phantom
    \sbox\myboxB{$\m@th\overline{\copy\myboxB}$}%  Overlined phantom
    \setlength\mylenA{\the\wd\myboxA}%   calc width diff
    \addtolength\mylenA{-\the\wd\myboxB}%
    \ifdim\wd\myboxB<\wd\myboxA%
       \rlap{\hskip 0.5\mylenA\usebox\myboxB}{\usebox\myboxA}%
    \else
        \hskip -0.5\mylenA\rlap{\usebox\myboxA}{\hskip 0.5\mylenA\usebox\myboxB}%
    \fi}
\makeatother

%%


